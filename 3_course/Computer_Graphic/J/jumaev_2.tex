\documentclass[11pt, a4paper]{article}
\usepackage[utf8]{inputenc}
\usepackage[russian]{babel}
\usepackage{amsmath}
\usepackage{amsfonts}
\usepackage{amssymb}

\textwidth=18cm
\oddsidemargin=-1cm
\evensidemargin=-1cm
\topmargin=-3cm
\textheight=32cm
\linespread{0.8}

\begin{document}
	\Large
	
	\bf УДК 517.91 \\
	\begin{center}
		Приложения одного класса дифференциальных\\
		уравнений к решению некоторых задач медицины
		\vspace{0.5cm}
		
		\rm Шарипов Б$^1$., Темуров З.А$^2$., Шоймкулов А.М$^1$., Орипов Т.С$^3$.
		\vspace{0.5cm}
		
		\itТаджикский государственный финансово-экономический университет$^1$\\
		Отделение урологии станции скорой помощи г. Душанбе, МЗРТ$^2$\\
		Денауский институт предпринимательства и педагогики РУ$^3$
		\vspace{0.5cm}
	\end{center}
	
	\bf Аннотация. \it Целью предлагаемого тезиса заключается в том, что решение задач по дифференциальным уравнениям в частных производных применяется в моделировании задач естественной науки, в медицине, биологии и других прикладных науках. Научная ценность работы состоит в том, что полученные в ней результаты расширяют и углубляют представления о построении   решений всех видов систем с вырождением уравнений.
	
	\bf Ключевые слова: \itсистемы дифференциальных уравнений, математические модели, единственное решение задачи, модель выздоровления человеческого организма, естественная наука, медицина.
	
	\begin{center}
		\bf The appendix of one class of the differential\\
		the equations to the decision of some problems of medicine
	\end{center}
	
	\bf The summary: \it The purpose of the offered thesis consists, that the decision of problems on the differential equations in private derivatives is applied in modeling of problems of a natural science, in medicine, biology and other applied sciences. Scientific value of work consists that the results received in it expand and deepen representations about construction of decisions of all kinds of systems with degeneration of the equations.
	
	\bf Keywords: \it systems of the differential equations, mathematical models, the unique decision of a problem, model the human body recover, a natural science, medicine.
	
	\rm В работах [1]-[3] были изложены приложения некоторых классов обыкновенных дифференциальных уравнений (ОДУ), и некоторые другие классы систем дифференциальных уравнений  в частных производных [4], имеющие  некоторые взаимосвязи к задачам экологии, биологии, медицины, и других отраслях естественной науки.
	
	\rm Основная задача предстоящего исследования в нашей исследуемой задаче, заключается  в том, что показать здесь каким образом изучаются популяция живых организмов реальной природы и  какова состояние человеческих организмов со времени рождения до смерти, а также  сохранении состояния  здоровья  человека  в взаимосвязь  с многими   факторами реальной  природы.
	
	\rm Необходимо заметить, чтобы состояния глаза, сердца, почки, желудка, а также другие части организма нормального взрослого человека сохраняющиеся до последних дней его жизни, играют важную роль в сохранении его здоровья. Если человек нарушает нормальные правила питания продуктов питания, занятия к различным видам спорту, то они отрицательно влияют на нормальную способность всей части его организма.
	
	\bf 1. \rm Пусть $y=y(x)$ представляет число особей в популяции здорового человека в момент времени $x$. Тогда, числом $a$ можем обозначать как количеством особей при рождающихся в популяции, а $b$ - как количеством особей, вымирающими, в течении тех моментов. Тогда скорость изменения числа указанных особей рождающийся и умирающих факторов, в момент времени популяции $x$, можно записать следующей формулой (см.[1])
	
	\vspace{4cm}
	
	\begin{equation} \tag{1}
		\frac{dy}{dx}=a-b;(0\leq,b\leq150 \hspace{0.2cm} \textit{лет}).
	\end{equation}
	
	\rm При этой постановки задачи 150 лет считается максимальная возраст человека, в настоящем периоде.
	Начальное условие, или задача Коши для уравнении (1) имеет вид:
	
	\begin{equation} \tag{2}
		y=y_0, \hspace{0.2cm} \text{при условии  } x=x_0.
	\end{equation}
	
	Заметим, что в решении предлагаемой задачи, разность особей $(a-b)$ постоянным неотрицательным числом, либо как функции зависящей от некоторой переменной $x$, либо от переменных $(x,y)$ - в линейных, либо нелинейных видах. Некоторые эти случаи, рассмотрим в отдельности;
	
	\bf a) \rm Пусть в модельной уравнении (1) имеет место $a\geq b$. Тогда больные люди для проверки состояния своего здоровья обращаются к терапевтам, кардиологам, урологам, окулистам, хирургам и другим врачам, по необходимым специальностей для выхода из своего больного положения.
	
	В результате интегрирования обыкновенного дифференциального уравнения (1), с начальным условием (2), будем иметь единственного ее решения в виде:
	\begin{equation*}
		y(x)=y_0+(a-b)\cdot(x-x_0),
	\end{equation*}
	где оно считается моделью выздоровления пациента, в некотором периоде.
	
	\noindent \rule{\textwidth}{0.5pt}
	\bf б) \rm Если разность $(a-b)$ - представляется как линейной функции, т.е. по каждым факторам, то модельное дифференциальное уравнение (1) может принимать вид:
	
	\begin{equation} \tag{3}
	\frac{dy}{dx}=(a-b)\cdot x\cdot y, (a>b),\text{либо}\frac{dy}{dx}=\frac{a-b}{x}\cdot y, \hspace*{0.3cm}\text{и другими моделями ОДУ}
	\end{equation}
	Интегрируя первого уравнения из (3), математическими методами, будем иметь единственного ее решения в следующем виде модели выздоровлении:
	
	\begin{equation} \tag{4}
		y(x)=y_0\cdot exp\left\{\frac{a-b}{2}\cdot(x^2-x_0^2) \right\}(a>b).
	\end{equation}
	Аналогично с первым случаем, в процессе интегрирования второго 
	уравнения из (3), находим следующее функциональное  соотношение, являющиеся другим моделям поставленной задачи следующей формулой:
	
	\begin{equation} \tag{5}
		y(x)=y_0\cdot(x^{a-b}-x_0^{a-b}).
	\end{equation}
	Учитывая формулы вида (4), (5) заметим, что они представляются как  математической модели выздоровлении состояние заболевшего человека - в различных направлениях медицины.
	
	\bf в) \rm Пусть $a,b$ являются особи, но не как постоянные числа, а непрерывные функции зависящие от переменной $x$, либо $(x,y)$ как: $a=a(x),b=b(x)$; либо вида $a=a(x,y),b=b(x,y)$, и других видов функций от одного, либо многих независимых переменных, так например в линейном или в нелинейном виде:
	
	\begin{equation} \tag{6}
		\frac{dy}{dx}=a(x)-b(x) \hspace{0.5cm}\text{либо }\frac{dy}{dx}=[a(x)-p(x)]*y^k-[b(x)-q(x)]\cdot y.
	\end{equation}
	Тогда в результате интегрирования каждых этих дифференциальных уравнений из (6), соответственно получаем:
	
	\vspace{5cm}
	
	\begin{equation} \tag{7}
		y(x)=y_0+A(x)-B(x), y(x)=exp[A(x)-B(x)]\cdot\left[y_0^{1-k}+(1-k)H(x,y)\right]^{1/(k-1)}.
	\end{equation}
	Пусть модель некоторой задачи анатомии людей, представляются в видах системы дифференциальных уравнений в частных производных первого порядка, одной неизвестной функции двух действительных переменных, с регулярными правыми частями, например вида:
	
	\begin{equation} \tag{8}
		\frac{\partial u}{\partial x}=a(x,y)\cdot u,\hspace{0.2cm}\frac{\partial u}{\partial y}=b(x,y)\cdot u,\text{ либо } \frac{\partial u}{\partial x}=a(x,y)\cdot m(u),\hspace{0.2cm}\frac{\partial u}{\partial y}=b(x,y)\cdot m(u).
	\end{equation}
	Для которых начальные условия, или задача Коши для этих и последующих п.д.- систем будет иметь вид: $u=u_0$, при условии $x=x_0$, $y=y_0$.
	
	\begin{equation} \tag{9}
		\frac{\partial u}{\partial x}=[a(x,y)-f(x,y)]\cdot m(u), \hspace{0.3cm}\frac{\partial u}{\partial y}=[b(x,y)-g(x,y)]\cdot m(u).
	\end{equation}
	$a,b,f,g,m(u)\in C^1(\overline{D}),\hspace{0.2cm}u=u(x,y)\in C^2(\overline{D}\times R^1),\hspace{0.2cm}0\leq a,b,f,g\leq C=const,\hspace{0.2cm}0\leq u\leq k.$ В последней формуле  зафиксировано число $k>0$. По нашему мнению число $k$ определяет уровень жизни  пациента по состоянию его, либо ее здоровья. Приравнивая смешанные производные второго порядка, получаем соотношения, где ее называем условием совместности  соответствующих систем (8) - (9). Когда условие совместности  указанных систем выполняются по всем переменным, тогда многообразия решений, либо единственного решения этих систем, удовлетворяющиеся задачи Коши, определяются непрерывными функциями,  во всей данной области вида:
	
	\begin{equation} \tag{10}
		u(x,y)=C\cdot exp\left\{A(x,y)-B(0,y)\right\},\text{ либо }u(x,y)=u_0\cdot exp\left\{A(x,y)-B(0,y)\right\}.
	\end{equation}
	
	\begin{equation} \tag{11}
		u(x,y)=exp\left\{A(x,y)-B(0,y)\right\}\cdot\left[u_0+H(x,y)\right]^{1/(1-k)}.
	\end{equation}
	
	\bf Замечание. \rm Если в решении предыдущих задач прибавить,   нарушения  состояния глаз (как глаукомы или катаракты), сердца, почках, состояние состава крови, тахикардии,  органов дыхания, желудка, и других частей органов человека, то получаем, что решая модельных дифференциальных уравнений, получаем, что (для органов человечества), следует решить систему дифференциальных уравнений с частными  производными, с сингулярными коэффициентами  для функций двух и более независимых переменных вида:
	
	\begin{equation} \tag{12}
		\frac{\partial u}{\partial x}=\frac{a(x,y)}{(x-x_0)(y-y_0)}\cdot m(u),\hspace{0.2cm}\frac{\partial u}{\partial y}=\frac{b(x,y)}{(x-x_0)(y-y_0)}\cdot n(u).
	\end{equation}
	
	\begin{equation} \tag{13}
		\begin{cases} 
			\frac{\partial u}{\partial x}=\frac{a(x,y,z)}{(x-x_0)(y-y_0)(z-z_0)}\cdot m(u),\hspace{0.2cm}\frac{\partial u}{\partial y}=\frac{b(x,y,z)}{(x-x_0)(y-y_0)(z-z_0)}\cdot m(u),\\
			\frac{\partial u}{\partial z}=\frac{b(x,y,z)}{(x-x_0)(y-y_0)(z-z_0)}\cdot m(u).
		\end{cases}
	\end{equation}
	
	\begin{equation} \tag{14}
		\frac{\partial u}{\partial x_k}=\frac{a(x_k)}{\prod\limits_{k=1}^{n}(x_k-x_k^{(0)})}
	\end{equation}
	
	\vspace{7cm}
	
	и других  видов  систем, с одним, либо со многими сингулярными факторами. В этих случаях люди страдают от  одного, либо многих видов болезни  частей организмов. Бывают и другие случаи, когда некоторые люди одновременно болеют различными видами болезнями, как урологические, простатит, аденомы, либо осколочными ранениями (в военных периодах) и другими. С точки зрения математики, каждые виды болезни, понимаются как сингулярность в некоторой части организма человека. Решая эти системы, математическими методами, создадим модели выздоровления больных людей, с полноценным  здоровым организмом.
	
	При этом, в последней модели $u_0$ - является определённое постоянное число как год жизни (максимум 150 лет), а $\omega(x,y)$ - известная непрерывная функция, числа $a_0,b_0>0$. Если в последней модели, как решение системы уравнений   (11) $x=x_0,y=y_0$, то $u(x,y)=0$, то это означает, что состояние здоровья пациента ухудшается, и пациент гибнет, либо его состояния совсем ухудшается. Аналогичным образом, интегрируя модельных систем уравнений (12)-(14), в случае тождественного выполнения их условий совместности, получим модель состояния здоровья  каждого больного, после лечения по всем частям организма, и выздоровление клетки организма больного против состояния болезни.
	
	\bf Вывод: \rm Исследуя указанные дифференциальные уравнения, получаем, что для ОДУ первого порядка, а также систем двух дифференциальных уравнений, и трёх уравнений, в случае выполнения их условий совместности, находим единственное решение  систем. Причём эти полученные формулы, или модели, как решения системы дифференциальных уравнений, являются некоторыми моделями выздоровления лечение глаз, зубов, сердце, почек, простаты и других частей организма пациентов.
	
	\begin{center}
		\bf Список цитируемых литературы
	\end{center}
	1.	Амелькин В.В. Дифференциальные уравнения в приложениях.- М.: наука, ГРФ-МЛ, 1987, 160 с.\\
	2.	Михайлов Л.Г. Некоторые переопределённые системы уравнений в  
	частных производных, с двумя неизвестными функциями. Душанбе,-116 с.\\
	3.	С.В.Фомин, М.Е.Беркинблит. Математические проблемы в биологии. Наука, ГРФМЛ, Москва, 1979.-198 с.\\
	4.	Юсупова З.Х., Шарипов Б. Приложения одного класса дифференциальных уравнений в некоторых задач медицины. Вестник науки, 2024,№480,международная научно-практическая конференция, Уфа (РФ). Сборник научных статьей.-  с. 7-16.
	
\end{document}
